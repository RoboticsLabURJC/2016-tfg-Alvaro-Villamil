\chapter{Introducción}
\label{ch:Introduccion}

En este proyecto se construirán escenarios para la enseñanza utilizando diversas tecnologías. Para poder saber cual es el marco de desarrollo y donde se engloban los conceptos utilizados vamos a introducir algunos términos que servirán como contexto y que nos permitirán entender mejor de dónde surje la necesidad de estos escenarios y cuall es la utilidad de los mismos.

\section{Robótica}
\label{sec:intr_robotica}

Según la Wikipedia\cite{wikipedia}, \textit{La robótica es la rama de la ingeniería mecatrónica, de la ingeniería eléctrica, de la ingeniería electrónica, de la ingeniería mecánica, de la ingeniería biomédica y de las ciencias de la computación que se ocupa del diseño, construcción, operación, disposición estructural, manufactura y aplicación de los robots.}

Otra definición menos técnica podría ser: La robótica es una ciencia o rama de la tecnología, que estudia el diseño y construcción de máquinas capaces de desempeñar tareas realizadas por el ser humano o que requieren del uso de inteligencia. Por tanto, estamos ante una ciencia que se encarga de diseñar máquinas que sean capaces de reemplazar al ser humano en algunas acciones. Es una disciplina con sus propios problemas, sus fundamentos y sus leyes, y podemos observar dos vertientes de la misma, la teórica y la práctica. En la parte teórica podemos agrupar todas las aportaciones de la informática, la intenligencia artificial y la automatización. En cuanto a la parte práctica observamos aportaciones relativas tanto a la construcción del robot como a la de su gestión, siendo por tanto destacadas las aportaciones de la mecánica, electrónica, programación, etc... Esto hace de la robótica una ciencia con un marcado carácter interdisciplinario.

\subsection{Historia de la robótica}
\label{subsec:intr_historiarobotica}

La historia de la robótica ha estado unida a la construcción de “artefactos”, que trataban de materializar el deseo humano de crear seres semejantes a nosotros que nos descargasen del trabajo. Desde los primeros pasos de la civilización el ser humano ha desarrollado ingenios tanto para facilitar sus labores como para imitar a la naturaleza, fascinando a sus congéneres. De los antiguos egipcios se conservan descripciones de más de 100 máquinas y autómatas, incluyendo un artefacto con fuego, un órgano de viento, una máquina operada mediante una moneda, una máquina de vapor, en la obra \textit{Pneumática y Autómata} de Herón de Alejandría. Los griegos nos dejaron creaciones como un pájaro de madera, a vapor, que fue capaz de volar, y genios como Leonardo da Vinci el diseño de un \textit{Caballero mecánico}. 

El francés Jacques de Vaucanson creó en el siglo XVIII lo que se consideran los primeros robots de la historia: El flautista, El tamborilero y el Pato con aparato digestivo. También creó el primer telar completamente automático del mundo. Unos años más tarde, Joseph Jacquard inventa en 1801 una máquina textil programable mediante tarjetas perforadas, y Henri Maillardert construyó una muñeca mecánica que era capaz de hacer dibujos. Al mismo tiempo, en Japón, se creaban juguetes mecánicos que sirven té, disparan flechas y pintan. Podemos considerar esta época el nacimiento de los robots tal como los conocemos actualmente.

En el año 1921 un escritor checo, Karel Capek, publica su obra \textit{“Los Robots Universales de Rossum”}, en la que aparece por primera vez la palabra “robot“ derivada de la palabra checa robota, que significa servidumbre o trabajo forzado. Unos años más tarde, en 1942, la revista americana \textit{Astounding Science Fiction} pública "Círculo Vicioso" (Runaround en inglés), una historia de ciencia ficción escrita por Isaac Asimov donde aparecen por primera vez las Tres leyes de la robótica. Estas leyes establecen lo siguiente:
\begin{quote}
\begin{enumerate}[1.ª]
	\item Un robot no hará daño a un ser humano o, por inacción, permitir que un ser humano sufra daño.
	\item Un robot debe hacer o realizar las órdenes dadas por los seres humanos, excepto si estas órdenes entrasen en conflicto con la 1ª Ley.
	\item Un robot debe proteger su propia existencia en la medida en que esta protección no entre en conflicto con la 1ª o la 2ª Ley.
\end{enumerate}
\end{quote}


Pese a que son fruto de una obra de ciencia ficción, estas leyes han dado la vuelta al mundo y multitud de científicos e investigadores las toman como ciertas, siendo un concepto que aún hoy tiene sentido para los futuros desarrollos en torno a sistemas autónomos. La autonomía de las máquinas debería acompañarse de medidas de seguridad que evitaran el daño a las personas. Esto es un precepto que las tres leyes de la robótica de Asimov contienen. De hecho, la idea del escritor era proteger al ser humano, que los robots, por muy avanzados que estuvieran, no pudieran volverse contra las personas. En el caso de un coche autónomo, si este conduce sin pasajeros dentro y va a chocar contra otro donde viajan varias personas, ¿debe el primer vehículo echarse a un lado aunque esté circulando correctamente y vaya a sufrir más daños si lo hace? La primera ley de Asimov diría que sí. En cuanto a la segunda ley, actualmente no se concibe el desarrollo de ningún sistema autónomo sin mecanismos que permitan a las personas manejarlos manualmente, y se considera que aún no existe un sistema tan fiable y preciso como para darle mayor autoridad que al ser humano que lo controla. Evidentemente un sistema autónomo hará todo lo posible para no sufrir daños. Como toda tecnología, está diseñada para que funcione y para que mantenga su integridad y su funcionamiento. Su duración será mayor o menor dependiendo de la calidad, pero desde luego no acometerá operaciones destinadas a estropearse. En el año 1982 Isaac Asimov publicó \textit{El robot completo} (The Complete Robot en inglés), una colección de cuentos de ciencia ficción escritos entre 1940 y 1976. En esta colección vuelve a explicar las tres leyes de la robótica con más ahínco y complejidad moral. Incluso llega a plantear la muerte de un ser humano por la mano de un robot con las tres leyes programadas, por lo que decide incluir una cuarta ley "La ley 0 (cero)": \textit{Un robot no hará daño a la Humanidad o, por inacción, permitir que la Humanidad sufra daño}.

En la década de 1950 se comienzan a desarrollar los primeros robots comerciales, nacidos de las patentes de la década anterior. Debido a la investigación en inteligencia artificial se encontraron maneras de emular el procesamiento de información humana con computadoras electrónicas y se desarrollaron una variedad de mecanismos para probar estas teorías. En 1956 se comercializa el primer robot de la compañía Unimation, fundada por George Devol y Joseph Engelberger. En 1961 se instala en una fábrica de la Ford Motors Company uno de estos robots, cuya función era la de levantar y apilar grandes piezas de metal caliente. En 1971 el “Standford Arm“, un pequeño brazo robótico de accionamiento eléctrico, se desarrolló en la Standford University. En 1973 la empresa Kuka\footnote{\url{https://www.kuka.com/}} desarrolla el primer robot industrial con seis ejes electromecánicos, el Famulus. Unos años más tarde, en 1975, la empresa Unimation comercializó un brazo manipulador programable universal llamado PUMA.

Pero la robótica no se basa sólo en las máquinas que revolucionaron los procesos industriales, el concepto de robótica incluye y cada vez se orienta más hacia los sistemas móviles autónomos, que son aquellos capaces de desenvolverse por sí mismos en entornos desconocidos sin necesidad de supervisión. Con esta idea en mente, en los setenta, la NASA inicio un programa de cooperación con el Jet Propulsión Laboratory para desarrollar plataformas capaces de explorar terrenos hostiles. El primer fruto de esta alianza sería el MARS-ROVER, que estaba equipado con un brazo mecánico tipo STANFORD, un dispositivo telemétrico láser, cámaras estéreo y sensores de proximidad. Sin embargo, fue la Unión Soviética en 1971 la primera en aterrizar un robot en la superficie de marte con éxito, el Mars 3, aunque la comunicación se perdió minutos después. No fue hasta 1976 que la NASA hizo llegar al primer robot estadounidense a Marte, el Viking I.

Desde entonces la robótica ha experimentado en multitud de aplicaciones y formatos con modelos sumamente ambiciosos, como es el caso del IT, diseñado para expresar emociones, el COG, tambien conocido como el robot de cuatro sentidos, el famoso SOUJOURNER o el LUNAR ROVER, vehículo de turismo con control remotos, y otros mucho mas específicos como el CYPHER, un helicóptero robot de uso militar, el guardia de trafico japonés ANZEN TARO o los robots mascotas de Sony. En el campo de los robots antropomorfos (androides) se debe mencionar el P3 de Honda que mide 1.60m, pesa 130 Kg y es capaz de subir y bajar escaleras, abrir puertas, pulsar interruptores y empujar vehículos, así como el robot ASIMO de la misma compañía, capaz de desplazarse de forma bípeda e interactuar con las personas.

En general la historia de la robótica la podemos clasificar en cinco generaciones (división hecha por Michael Cancel, director del Centro de Aplicaciones Robóticas de Science Application Inc. en 1984): 
\begin{enumerate}[1.ª Generación]
	\item[1.ª Generación] Robots manipuladores. Son sistemas mecánicos multifuncionales con un sencillo sistema de control, bien manual, de secuencia fija o de secuencia variable.
	\item[2.ª Generación] Robots de aprendizaje. Repiten una secuencia de movimientos que ha sido ejecutada previamente por un operador humano. El modo de hacerlo es a través de un dispositivo mecánico. El operador realiza los movimientos requeridos mientras el robot le sigue y los memoriza.
	\item[3.ª Generación] Robots con control sensorizado. El controlador es una computadora que ejecuta las órdenes de un programa y las envía al manipulador para que realice los movimientos necesarios.
	\item[4.ª Generación] Robots inteligentes. Son similares a los anteriores, pero además poseen sensores que envían información a la computadora de control sobre el estado del proceso. Esto permite una toma inteligente de decisiones y el control del proceso en tiempo real.
\end{enumerate}


Las dos primeras, ya fueron alcanzadas en los ochenta. La tercera generación, que incluye visión artificial, ha avanzado mucho en los ochenta y noventas. La cuarta contempla movilidad avanzada en exteriores e interiores. Y podríamos hablar incluso de una quinta, en la cual entrarían los más modernos sistemas de aprendizaje autónomo y la inteligencia artificial.

Otra clasificación muy extendida de los robots es según su estructura. La estructura está definida por el tipo de configuración general del Robot. El concepto de metamorfismo, de reciente aparición, se ha introducido para incrementar la flexibilidad funcional de un Robot a través del cambio de su configuración por el propio Robot. El metamorfismo admite diversos niveles, desde los más elementales (cambio de herramienta o de efecto terminal), hasta los más complejos como el cambio o alteración de algunos de sus elementos o subsistemas estructurales. Los dispositivos y mecanismos que pueden agruparse bajo la denominación genérica del Robot, tal como se ha indicado, son muy diversos y es por tanto difícil establecer una clasificación coherente de los mismos que resista un análisis crítico y riguroso. La subdivisión de los Robots, con base en su arquitectura, se hace en los siguientes grupos: 
\begin{enumerate}[1.]
	\item Poliarticulados. En este grupo se encuentran los Robots de muy diversa forma y configuración, cuya característica común es la de ser básicamente sedentarios (aunque excepcionalmente pueden ser guiados para efectuar desplazamientos limitados) y estar estructurados para mover sus elementos terminales en un determinado espacio de trabajo según uno o más sistemas de coordenadas, y con un número limitado de grados de libertad. En este grupo, se encuentran los manipuladores, los Robots industriales, los Robots cartesianos y se emplean cuando es preciso abarcar una zona de trabajo relativamente amplia o alargada, actuar sobre objetos con un plano de simetría vertical o reducir el espacio ocupado en el suelo.
	\item Móviles. Son Robots con gran capacidad de desplazamiento, basados en carros o plataformas y dotados de un sistema locomotor de tipo rodante. Siguen su camino por telemando o guiándose por la información recibida de su entorno a través de sus sensores. Estos Robots aseguran el transporte de piezas de un punto a otro de una cadena de fabricación. Guiados mediante pistas materializadas a través de la radiación electromagnética de circuitos empotrados en el suelo, o a través de bandas detectadas fotoeléctricamente, pueden incluso llegar a sortear obstáculos y están dotados de un nivel relativamente elevado de inteligencia.
	\item Androides. Son los tipos de Robots que intentan reproducir total o parcialmente la forma y el comportamiento cinemático del ser humano. Actualmente, los androides son todavía dispositivos muy poco evolucionados y sin utilidad práctica, y destinados, fundamentalmente, al estudio y experimentación. Uno de los aspectos más complejos de estos Robots, y sobre el que se centra la mayoría de los trabajos, es el de la locomoción bípeda. En este caso, el principal problema es controlar dinámica y coordinadamente en el tiempo real el proceso y mantener simultáneamente el equilibrio del Robot.
	\item Zoomórficos. Los Robots zoomórficos, que considerados en sentido no restrictivo podrían incluir también a los androides, constituyen una clase caracterizada principalmente por sus sistemas de locomoción que imitan a los diversos seres vivos. A pesar de la disparidad morfológica de sus posibles sistemas de locomoción es conveniente agrupar a los Robots zoomórficos en dos categorías principales: caminadores y no caminadores. El grupo de los Robots zoomórficos no caminadores está muy poco evolucionado. Los experimentos efectuados en Japón basados en segmentos cilíndricos biselados acoplados axialmente entre sí y dotados de un movimiento relativo de rotación. Los Robots zoomórficos caminadores multípedos son muy numerosos y están siendo objeto de experimentos en diversos laboratorios con vistas al desarrollo posterior de verdaderos vehículos terrenos, pilotados o autónomos, capaces de evolucionar en superficies muy accidentadas. Las aplicaciones de estos Robots serán interesantes en el campo de la exploración espacial y en el estudio de los volcanes.
	\item Híbridos.	Estos Robots corresponden a aquellos de difícil clasificación, cuya estructura se sitúa en combinación con alguna de las anteriores ya expuestas, bien sea por conjunción o por yuxtaposición. Por ejemplo, un dispositivo segmentado articulado y con ruedas, es al mismo tiempo, uno de los atributos de los Robots móviles y de los Robots zoomórficos.
\end{enumerate}





